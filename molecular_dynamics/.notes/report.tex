%% LyX 2.3.6.1 created this file.  For more info, see http://www.lyx.org/.
%% Do not edit unless you really know what you are doing.
\documentclass[british]{article}
\usepackage[T1]{fontenc}
\usepackage[latin9]{inputenc}
\usepackage{geometry}
\geometry{verbose,tmargin=2cm,bmargin=2cm,lmargin=1cm,rmargin=1cm}
\usepackage{float}
\usepackage{amsmath}
\usepackage{amsthm}
\usepackage{setspace}
\doublespacing

\makeatletter
%%%%%%%%%%%%%%%%%%%%%%%%%%%%%% User specified LaTeX commands.
\usepackage{indentfirst}
\usepackage{mathtools}

\makeatother

\usepackage{babel}
\begin{document}
\title{Molecular Dynamics}
\author{Leonardo H�gens and Johanna L�mker}
\maketitle

\section{Methods}

Molecular Dynamics simulations consist in numerically integrating
the equations of motion of a system, starting from an initial configuration,
and measuring time averages of quantities of interest.

\subsection{Initial Conditions}

Our system consists of $N$ particles in a cubic box $\left[0,L\right]^{3}$.
The initial positions where chosen at random with uniform probability
inside the box.

For initial velocities, we first generate every velocity component
randomly with uniform distribution in the interval $\left[-1,1\right]$.
We then calculate the total momentum $\boldsymbol{P}$, and shift
all the velocities equally such that the new total momentum is $0$.
Then, we rescale every velocity component by the same constant such
that the total kinetic energy matches the desired temperature at which
we want to perform our simulation:
\[
E_{kin}=\sum_{i=1}^{N}\frac{mv_{i}^{2}}{2}=\frac{3Nk_{B}T}{2}.
\]


\subsection{Particle Interaction}

We consider the Lennard-Jones potential energy between every pair
of particles, given by:
\[
u\left(r_{ij}\right)=\left\{ \begin{array}{ll}
4\epsilon\left[\left(\frac{\sigma}{r_{ij}}\right)^{12}-\left(\frac{\sigma}{r_{ij}}\right)^{6}\right]-e_{\mathrm{cut}} & r_{ij}\leq r_{\mathrm{cut}}\\
0 & r_{ij}>r_{\mathrm{cut}}
\end{array}\right.,
\]

where
\[
e_{\mathrm{cut}}=4\epsilon\left[\left(\frac{\sigma}{r_{\mathrm{cut}}}\right)^{12}-\left(\frac{\sigma}{r_{\mathrm{cut}}}\right)^{6}\right]
\]

and $r_{\mathrm{cut}}$ is the maximum distance two particles can
have from each other to still be affected by each other's potential.
To obtain the force, we determine the gradient of this potential,
which gives:
\[
\boldsymbol{f}_{ij}=-\nabla u=-48\frac{\boldsymbol{r}_{ij}}{r_{ij}^{8}}\left(\frac{1}{r_{ij}^{6}}-\frac{1}{2}\right)
\]


\subsection{Integration Scheme}

The integration scheme we used in this assignment is the \emph{Velocity
Verlet} algorithm, whose recursive process for each particle goes
as follows, given it's position $\boldsymbol{r}\left(t\right)$, velocity
$\boldsymbol{v}\left(t\right)$ and force acting on it $\boldsymbol{f}(t)$:
\begin{align*}
\boldsymbol{r}(t+dt) & =\boldsymbol{r}\left(t\right)+\boldsymbol{v}\left(t\right)dt+\frac{\boldsymbol{f}(t)}{2m}dt^{2}\\
\boldsymbol{v}\left(t+dt\right) & =\boldsymbol{v}\left(t\right)+\frac{1}{2m}\left(\boldsymbol{f}(t)+\boldsymbol{f}(t+dt)\right)
\end{align*}

We also apply periodic boundary conditions, not only implicitly in
the calculation of the force but also explicitly in the position updates
(apply them when a particles falls outside the box), such that all
positions are always inside the box.

\subsection{NVE and NVT}

For the NVE ensemble, we just run the simulation as described above,
storing $E_{kin}$and $E_{potential}$ at every time step. For the
NVT ensemble, we define a frequency $\nu$, and at every time step
select each particle with probability $\nu\,dt$ to `undergo a collision
with a heat bath at temperature $T$', which in practice means setting
its component's values to randomly generated ones according to the
Maxwell-Boltzmann distribution:
\[
f\left(v_{i}\right)=\sqrt{\frac{m}{2}\frac{\beta}{\pi}}e^{-\frac{m}{2}\beta v_{i}^{2}},
\]

which is a gaussian distribution with mean $\mu=0$ and standard deviation
$\sigma=\frac{1}{\beta m}$.
\end{document}
