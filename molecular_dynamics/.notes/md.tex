%% LyX 2.3.6.1 created this file.  For more info, see http://www.lyx.org/.
%% Do not edit unless you really know what you are doing.
\documentclass[british]{article}
\usepackage[T1]{fontenc}
\usepackage[latin9]{inputenc}
\usepackage{geometry}
\geometry{verbose,tmargin=2cm,bmargin=2cm,lmargin=1cm,rmargin=1cm}
\usepackage{float}
\usepackage{amsmath}
\usepackage{amsthm}
\usepackage{setspace}
\doublespacing

\makeatletter
%%%%%%%%%%%%%%%%%%%%%%%%%%%%%% User specified LaTeX commands.
\usepackage{indentfirst}
\usepackage{mathtools}

\makeatother

\usepackage{babel}
\begin{document}
\title{Notes}
\author{Leonardo H�gens\\
l.lobatodiasleitehugens@students.uu.nl}
\maketitle

\subsection*{Verlet Algorithm}

The original Verlet Algorithm is
\[
r(t+\Delta t)=2r(t)-r(t-\Delta t)+\frac{f(t)}{m}\Delta t^{2}.
\]

This brings up a problem with boundary conditions. It is perfectly
possible that $r(t)$ and $r(t-\Delta t)$ but $r(t+\Delta t)$ falls
outside the box. After applying boundary conditions, the new $r(t+\Delta t)$
is typically far away from the original one. In the new iteration,
this new position will correspond to $r(t)$, and it will be very
far away from the $r(t-\Delta t)$ of this new iteration, producing
unintentional big jumps. Thus, we rewrote this algorithm in terms
of a displacement $dr(t,t-\Delta t)=r(t)-r(t-\Delta t)$, which gives:
\begin{align*}
r(t+\Delta t) & =r(t)+dr(t,t-\Delta t)+\frac{f(t)}{m}\Delta t^{2}\\
dr(t+\Delta t,t) & =dr(t,t-\Delta t)+\frac{f(t)}{m}\Delta t^{2}
\end{align*}

Thus, instead of always storing two consecutive positions, we store
the last position and the last displacement: update the displacement
using the equation above and adding that displacement to the last
position to get the new position.

\subsection*{Lennard-Jones Interaction}

The Lennard-Jones potential we are considering is:
\[
u\left(r_{ij}\right)=\left\{ \begin{array}{ll}
4\epsilon\left[\left(\frac{\sigma}{r_{ij}}\right)^{12}-\left(\frac{\sigma}{r_{ij}}\right)^{6}\right]-e_{\mathrm{cut}} & r_{ij}\leq r_{\mathrm{cut}}\\
0 & r_{ij}>r_{\mathrm{cut}}
\end{array}\right.
\]

where
\[
e_{\mathrm{cut}}=4\epsilon\left[\left(\frac{\sigma}{r_{\mathrm{cut}}}\right)^{12}-\left(\frac{\sigma}{r_{\mathrm{cut}}}\right)^{6}\right]
\]

To obtain the force, we can just calculate the gradient of this potential.
Denoting $\boldsymbol{r}_{ij}=\boldsymbol{r}_{i}-\boldsymbol{r}_{j}$,
this gives:
\[
\boldsymbol{f}_{ij}=48\frac{\boldsymbol{r}_{ij}}{r_{ij}^{8}}\left(\frac{1}{r_{ij}^{6}}-\frac{1}{2}\right)
\]


\subsection*{Andersen Thermostat}

We want to make an evolution of the system such that it conserves
the temperature, which can be obmtain by the total kinetic energy
by:

\[
E_{kin}=\sum_{i=1}^{N}\frac{mv_{i}^{2}}{2}=\frac{3Nk_{B}T}{2}.
\]

In our case, since $N$ is constant, conserving temperature happens
if and only if kinetic energy conservation happens.

To implement the Andersen thermostat, at each time step, each particle
has a probability of $\nu\Delta t$ of being selected to receive a
new velocity, drawn from the Maxwell-Boltzmann distribution at the
desired temperature. So $\nu$ is the frequency of stochastic collisions
(the higher, the more couple the system is to the `heat bath').

The Maxwell-Boltzmann distribution for each component of the velocity
is given by:
\[
f\left(v_{i}\right)=\sqrt{\frac{m}{2}\frac{\beta}{\pi}}e^{-\frac{m}{2}\beta v_{i}^{2}},
\]

which has mean $\mu=0$ and standard deviation
\end{document}
